\subsection{Model \#2: fixed planets and primaries, varying secondaries}
\label{sec:model_2}

The main use of our binary-twin model is to help develop intuition.
We now let the light ratio $\gamma_R = L_2/L_1$ vary across the binary 
population.
It does so because the underlying mass ratio $q=M_2/M_1$ varies.
We keep the primary mass fixed as $M_1$, which is also the mass of all single 
stars.

We parametrize the distribution of binary mass ratios in a volume-limited 
sample as a power law: $p(q)\propto q^\beta$.
For binaries with solar-type primaries\footnote{
Duchene and Kraus (2013), fitting all the multiple systems of Raghavan et al. 
(2010)'s Fig 16, find $\beta = 0.28\pm0.05$ for $0.7<M_\star/M_\odot<1.3$.
Examining only the binary systems of Rhagavan et al 2010, Fig 16, the 
distribution seems roughly uniform, $\beta \approx 0$, except for a claimed 
excess of twin binaries with $q\approx 1$, and an obvious lack of $q<0.1$ 
stellar companions.
}, $\beta$ is probably between 0 and 0.3.
Since we assume stars are a one-parameter family, $R \propto M \propto 
L^{1/\alpha}$, a drawn value of $q$ determines everything about a secondary.

The rate density in this model, $\Gamma(\vec{x})$, is the sum over system 
types of $w_i \Lambda_i p_i(\vec{x})$:
\begin{equation}
\Gamma(\vec{x})
=
\delta^4(r_p,R_\star,a_p,P_p)(w_0 \Lambda_0 + w_1 \Lambda_1)
+ w_2 \Lambda_2 \delta^3(r_p, P_p, a_p) p_2(q),
\label{eq:model2_rate_density}
\end{equation}
where the semimajor axis of the planet must be such that its period is $P_p$, 
and $p_2(q)$ is expressed in terms of the mass ratio instead of the secondary 
star's radius for convenience ($q$ and $R_2$ are interchangeable).
The probability that a secondary hosts a planet, as a function of the mass 
ratio, is
\begin{align}
p_2(q) &= p({\rm has\ planet}\,|\,{\rm secondary\ with\ }q) \times
          p({\rm secondary\ with\ }q)
          \\
p_2(q) &\propto q^{\gamma + \beta} (1+q^\alpha)^{3/2}.
\label{eq:model_2_p_2}
\end{align}
We take first term, $p({\rm has\ planet}\,|\,{\rm secondary\ with\ }q)$, as a 
power law of $q$ with exponent $\gamma$.
For the second term, since the selected sample at a given $(r,P,a)$ is
magnitude-limited, $p({\rm secondary\ with\ }q)$ 
is the product of the volume limited probability, $q^\beta$, and a 
Malmquist-like bias, $(1+q^\alpha)^{3/2}$.
We note that various authors (ourselves included) have incorrectly 
used volume-limited binary distributions in Monte Carlo simulations of 
transit surveys (Sullivan et al 2015, Hartman and Bakos XX, Guenther et al 
2017, Bouma et al 2017).

The occurrence rate corresponding to Eq.~\ref{eq:model2_rate_density}'s rate 
density for a desired volume of phase space $\Omega_{\rm desired}$ is given by 
Eq.~\ref{eq:occ_rate}.
Specifying the desired mass ratios of interest as $q_{\rm min} < q < q_{\rm 
max}$, this simplifies to
\begin{equation}
\Lambda = 
\frac{N_0 \Lambda_0 + N_1 \Lambda_1 + N_2 \Lambda_2 f_2}
{N_{\rm tot}},
\end{equation}
for
\begin{equation}
f_2 \equiv
\left(
\int_{q_{\rm min}}^{q_{\rm max}} p_2(q) \,{\rm d}q
\right)
\cdot
\left(
\int_{0}^{1} p_2(q) \,{\rm d}q
\right)^{-1}.
\end{equation}

The detected rate density, $\hat{\Gamma} = \sum_i Q_i \Gamma_i$, will be 
specified by the detection efficiencies for each type of system.
These are identical to 
Eqs.~\ref{eq:detection_efficiency_0}-\ref{eq:detection_efficiency_2}.
The detection efficiency for secondaries (Eq.~\ref{eq:detection_efficiency_2}) 
includes the transit probability from the smaller stellar radius, and combines 
dilution, the transit duration, and stellar radius for the completeness
probability.

\paragraph{What does an observer ignoring binarity infer?}
As a reminder, the apparent rate density is found by correcting the detected 
apparent rate density for the transit probability:
$\Gamma_a = \tilde{\Gamma} Q_{g,0}^{-1}$.
The observer's errors are as follows:
\begin{enumerate}
    \item The true planetary radii $r$ are interpreted as apparent radii $r_a$.
    The apparent radii depend on the system type:
    \begin{align}
    r_a
    &=
    \left.
    \begin{cases}
    r_p (1+q^\alpha)^{-1/2} & \text{for } i=1,\ {\rm primary} \\
    r_p (1+q^{-\alpha})^{-1/2} q^{-1}, & \text{for } i=2,\ {\rm secondary}.
    \end{cases}
    \right.
    \label{eq:model2_ra}
    \end{align}
    The factor of $q^{-1}$ for the secondary case accounts for the observer 
    assuming that the host star is the primary.
    \item The completeness fraction is miscalculated for any stars in binary 
    systems.
    \item The selected and searchable stars are miscounted.
\end{enumerate}

To write the apparent rate density as a function of the apparent 
radius $r_a$, we marginalize out the planet period, semimajor axis, and 
stellar radius (or equivalently the mass ratio, for binaries):
\begin{equation}
\Gamma_a(r_a) =
w_a \Lambda_0 \delta(r_p)
+
w_b \Lambda_1 I_1(r_a)
+
w_b \Lambda_2 I_2(r_a),
\label{eq:model2_Gamma_a}
\end{equation}
where the detection efficiencies are given in 
Eqs.~\ref{eq:detection_efficiency_0}-\ref{eq:detection_efficiency_2}, and as 
in the first model, $w_a=N_0/(N_0+N_1)$, $w_b=N_1/(N_0+N_1)$. The ratio of 
primaries to singles, $\mu$, is now given by a variant of 
Eq.~\ref{eq:mu_definition}:
\begin{equation}
\mu \equiv \frac{N_1}{N_0} = \frac{\rm BF}{1 - {\rm BF}} \left(2^{3/2} - 
\int_{1}^{\sqrt{2}} u^2 (u^2 -1)^{1/\alpha} \,{\rm d}u\right),
\label{eq:mu_model_2}
\end{equation}
where the latter dimensionless integral is easily found numerically.
The $I_1(r_a)$ and $I_2(r_a)$ terms marginalize over the joint distribution of 
apparent radius and mass ratio:
\begin{align}
I_i(r_a) &= 
\int_0^1 p({\rm has\ detected\ planet}, r_a, q | {\rm star\ is\ type\ }i)
    \,{\rm d}q,
\quad
{\rm for}\ i\in\{1, 2\}, \\
&=
\int_0^1 
    p({\rm has\ detected\ planet} | r_a, q, {\rm star\ is\ type\ }i) 
    \nonumber\\
    &\quad\quad\quad\quad\times p(r_a | q, {\rm star\ is\ type\ }i)
    \,p(q | {\rm star\ is\ type\ }i)
\,{\rm d}q.
\end{align}
The first term is the detection efficiency; the second is a $\delta$-function 
of the apparent radius; the last is the mass ratio distribution given by 
Eq.~\ref{eq:model_2_p_2}.
The analytic solution for $i=1$ is
\begin{equation}
I_1(r_a) = \frac{1}{\mathcal{N}_1} \left(\frac{r_p}{r_a}\right)^{-3}
\left( \left(\frac{r_p}{r_a}\right)^2 -1  
    \right)^{\frac{\gamma+\beta}{\alpha}},
\quad {\rm for}\ r_p/\sqrt{2} < r_a < r_p,
\end{equation}
where $\mathcal{N}_1$ is the normalization term of the binary mass ratio 
distribution (Eq.~\ref{eq:model_2_p_2}): $\mathcal{N}_1 = \int_0^1 
q^{\gamma+\beta} (1+q^\alpha)^{3/2} {\rm d}q$.

For $i=2$, there is no analytic solution, because evaluating the integral 
requires imposing the constraint that $r_a  = r_p 
(1+q^{-\alpha})^{-1/2}q^{-1}$. This equation can be re-written
\begin{equation}
\left(\frac{r_p}{r_a}\right)^2 = q^2 + q^{-\alpha + 2},
\end{equation}
which has no analytic solution except for special values of $\alpha$, the 
mass-luminosity exponent.
For $\alpha=3.5$, our nominal case, semianalytic solutions can be found.

Since our main interest is in understanding the qualitative behavior 
of the solutions, we focus on a few analytic limiting cases, and then 
proceed numerically.


\paragraph{Correction to inferred rate density}
Recall that the rate density correction factor, $X_\Gamma$, is the ratio of 
the apparent to true rate densities.
We consider a ``nominal model'' in which the stellar population is similar to 
Sun-like stars in the local neighborhood:
${\rm BF}=0.44$, $\alpha=3.5$, $\beta=0$.
Our default assumption is also that the occurrence of planets is independent 
of stellar mass ($\gamma=0$), so secondaries have the same occurrence rate as 
primaries and single stars.
Under these assumptions, the planetary rate density is
\begin{equation}
\Gamma(r) \approx \delta(r_p) \left( \Lambda_0 + \Lambda_1 + 
\Lambda_2 \right) / 3,
\label{eq:model2_Gamma_r}
\end{equation}
where the coefficients of $1/3$ are accurate to within one percent of the true 
coefficients.
Ignoring binarity, the observer finds an apparent rate density
\begin{equation}
\Gamma_a(r) = c_0 \Lambda_0 \delta(r_p)
             +c_1 \Lambda_1 I_1(r_a)
             +c_2 \Lambda_2 I_2(r_a),
\label{eq:model2_Gamma_a_r}
\end{equation}
for $c_0\approx 0.49$, $c_1\approx 0.32$, $c_2\approx 0.03$.
%171011_integrals.nb
Evaluating the correction term at $r=r_p$, since $\lim_{q\rightarrow0} 
p_i(r_a)=0$ for $i\in\{1,2\}$,
\begin{equation}
X_\Gamma(r=r_p) \approx \frac{3c_0 \Lambda_0}{\Lambda_0+\Lambda_1+\Lambda_2}.
\end{equation}
If all the rates are equal, $X_\Gamma(r=r_p)\approx0.49$.
If there are no planets around the secondaries, $X_\Gamma(r=r_p)\approx0.74$.

We also run this model in our numerical simulation.
The results are shown in Figs.~\ref{fig:errcases_model_2_linear} 
and~\ref{fig:errcases_model_2_log}.
They produce the same correction factors as predicted analytically.
They also show a peak in the inferred rate at $r_p/\sqrt{2}$ from 
secondaries.
This effect requires the observer to incorrectly estimate the host star's 
radius; if they somehow knew the host star's radius, but did not correct for 
binary dilution, the peak would instead be at an apparent radius of 0.

To summarize, dilution produces a spectrum of apparent planetary radii. In 
this model, this produces overestimated rates everywhere except where there 
are actually planets, where the rate is underestimated by a factor of 2.