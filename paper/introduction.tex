\section{Introduction}

{\bf TODO} -- how should we open this? Should I include less on the HJ rate 
thing below?

\paragraph{Binarity and Occurrence Rates}



\paragraph{The Hot Jupiter Rate Discrepancy}

Radial velocity surveys of nearby bright Sun-like stars have reported hot 
Jupiter ($a<0.1\,{\rm AU}, P\lesssim 10\,{\rm day}$) occurrence rates of
$12\pm 1$, $15\pm 6$, $8.9 \pm 3.6$, and $12.0 \pm 3.8$ per thousand stars
(Marcy et al 2005, Cumming et al 2008, Mayor et al 2011, Wright et al 2012 
respectively).

In transit surveys, OGLE-III reported an absolute rate of $3.1^{+ 
4.3}_{-1.8}$ hot Jupiters with $P<5\,{\rm days}$ per thousand stars (Gould et 
al 2006).
The SuperLupus survey quoted $10^{+27}_{-8}$ per thousand stars, allowing 
planets with periods less than 10 days (Bayliss \& Sackett 2011).
Through the {\it Kepler}\ survey, Howard et al. (2012) later reported a rate 
of 
$4 \pm 1$ per thousand Sun-like stars.
This rate was for $P<10\,{\rm day}, 8-32r_\oplus$ planets, and 
was specific to their ``solar subset''\footnote{The ``solar subset'' was 
defined as $4100<T_{\rm eff}/{\rm K}<6100$, {\it Kepler}\ 
magnitude $<15$, $4.0 < \log g < 4.9$, and only took planets with measured 
signal to noise $>10$.
}.
Expanding their sample to fainter magnitudes, they quoted a rate of $5 \pm 
1$.
Expanding down to $r_p>5.6r_\oplus$, to avoid excluding hot Jupiters reported 
by RV surveys, they reported $7.6 \pm 1.3$.
Recent work by Petigura et al. using the updated parameters of the 
California-{\it Kepler}\ Survey has found a rate of $X.X \pm 
Y.Y$ (2017, in preparation)

The trend is that hot Jupiter occurrence rates measured by transit 
surveys seem to be marginally lower than those found by radial velocity 
surveys.
While the actual discrepancy is of sub-$3\sigma$ significance,
one reason to expect a difference in the rates is that the populations have 
different metallicities.
As originally argued by Gould et al. (2006), the RV sample is biased towards 
metal-rich stars, which have been measured by RV surveys to preferentially 
host more giant planets (Santos et al 2004, Fischer and Valenti 2005).
The {\it Kepler}\ sample specifically has been measured to be more metal poor 
than the local neighborhood, with a mean metallicity of $[{\rm M/H}]_{\rm 
mean}\approx -0.05$ (Dong et al., 2014; Guo et al., 2017).
Studying the problem in detail, Guo et al. recently argued that the 
metallicity difference could account for a $\approx 0.1\%$ difference in the 
measured rates between the CKS and {\it Kepler}\ samples~--~not a $\approx 
0.5\%$ difference.
These authors concluded that ``other factors, such as binary contamination and 
imperfect stellar properties'' must also be at play (Guo et al., 2017).

Aside from surveying stars of different metallicities, radial velocity and 
transit surveys differ in how they treat binarity.
Radial velocity surveys typically reject both visual and spectroscopic binaries
({\it e.g.}, Wright et al. 2012).
Transit surveys typically observe binaries, but the question of whether they 
were searchable to begin with is usually left for later interpretation.
In spectroscopic follow-up of transiting candidates, the prevalence of 
astrophysical false-positives may also lead to a bias against confirmation of 
transiting planets in binary systems.

Ignoring these complications, in this work we focus on whether
binarity might intrinsically bias occurrence rate measurements, simply 
through its effects on the number of searchable stars and the apparent radii 
of detected planets.

To progressively gain intuition, we study the following idealized transit 
surveys:
\begin{itemize}
    \item Model \#1: fixed stars, fixed planets, twin binaries;
    \item Model \#2: fixed planets and primaries, varying secondaries;
    \item Model \#3: fixed primaries, varying planets and secondaries.
\end{itemize}
In Sec.~\ref{sec:numerical_methods}, we introduce our numerical approach 
to the problem, and in Sec.~\ref{sec:analytic_preliminaries} we clarify our 
terminology.
We present the analytic and numerical results in 
Secs.~\ref{sec:model_1}-\ref{sec:model_3}, where each subsection corresponds 
to each model above.
We interpret these calculations throughout, and in 
Sec.~\ref{sec:discussion} discuss their relevance to various questions in 
the interpretation of transit survey occurrence rates. 