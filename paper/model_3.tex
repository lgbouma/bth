\subsection{Model \#3: Fixed primaries, varying planets and secondaries}
\label{sec:model_3}

In this model, as in the previous one, all single and primary stars have 
identical properties.
Only the secondaries have masses, radii, and luminosities that vary between 
systems: $M\propto R \propto L^{1/\alpha}$.
The radii of planets are assigned independently of any host system
property, and are sampled from an intrinsic radius distribution, which we take 
as
\begin{align}
p_r(r)
&\propto
\left.
\begin{cases}
r^\delta & \text{for } r\geq 2R_\oplus \\
{\rm constant} & \text{for } r\leq2R_\oplus.
\end{cases}
\right.
\label{eq:model3_radius_distribution}
\end{align}
Following Howard et al. (2012)'s measurement, we take $\delta = -2.92$.
Our ``nominal model'' remains the same: the binary fraction is 0.44, 
$\alpha=\beta=\gamma=0$.
We take $\Lambda_i$, the occurrence rate integrated over all possible phase 
space for the $i^{\rm th}$ system type, to be equal for singles, primaries, 
and secondaries. 

For this model, we forgo analytic development and focus only on numerics 
(Sec.~\ref{sec:numerical_methods}).
The nominal results are shown in Fig.~\ref{fig:errcases_model_3_log}.
For the assumed planetary and stellar distributions, the inferred rate is 
underestimated over all radii.
Taking Fig.~\ref{fig:errcases_model_3_log} and counting the number of planets 
per star with $r>8R_\oplus$, we can compare the true and inferred hot Jupiter 
occurrence rates.
Under the previously described assumptions, the true rate is 9.1 hot Jupiters 
per thousand stars.
The inferred rate is 6.9 per thousand stars.
This means that the inferred rate underestimates the true rate by a 
multiplicative factor of $\sim 1.3$.
This number is suggestively close to the claimed discrepancy between 
hot Jupiter occurrence rates measured by radial velocity and transit surveys 
(Wright et al. 2012).

However, this result only applies under the assumption that $\Lambda_0 = 
\Lambda_1 = \Lambda_2$.
Given that hot Jupiters are thought to be less common around less-massive 
stars, it would be more sensible to consider $\Lambda_2<\Lambda_0$, while 
letting single stars and primaries host planets at the same rate.
We show the HJ multiplicative correction factor as a function of 
$\Lambda_2/\Lambda_0$ in Fig BLAH.
