\subsection{Model \#3: Fixed primaries, varying planets and secondaries}
\label{sec:model_3}

In this model, as in the previous one, all single and primary stars have 
identical properties.
Only the secondaries have masses, radii, and luminosities that vary between 
systems: $M\propto R \propto L^{1/\alpha}$.
The radii of planets are assigned independently of any host system
property, and are sampled from an intrinsic radius distribution, which we take 
as
\begin{align}
p_r(r)
&\propto
\left.
\begin{cases}
r^\delta & \text{for } r\geq 2r_\oplus \\
{\rm constant} & \text{for } r\leq2r_\oplus.
\end{cases}
\right.
\label{eq:model3_radius_distribution}
\end{align}
Following Howard et al. (2012)'s measurement, we take $\delta = -2.92$.
Our ``nominal model'' remains the same: the binary fraction is 0.44, 
$\alpha=\beta=\gamma=0$.
We take $\Lambda_i$, the occurrence rate integrated over all possible phase 
space for the $i^{\rm th}$ system type, to be equal for singles, primaries, 
and secondaries. 

For this model, we forgo analytic development and focus only on numerics 
(Sec.~\ref{sec:numerical_methods}).
The nominal results are shown in Fig.~\ref{fig:errcases_model_3_log}.
For the assumed planetary and stellar distributions, the inferred rate is 
underestimated over all radii.

\paragraph{Hot Jupiter Occurrence Rates}
Taking Fig.~\ref{fig:errcases_model_3_log} and counting the number of planets 
per star with $r>8r_\oplus$, we can compare the true and inferred hot Jupiter 
occurrence rates.
Under the above assumptions, the true rate is 9.1 hot Jupiters 
per thousand stars.
The inferred rate is 6.9 per thousand stars.
This means that the inferred rate underestimates the true rate by a 
multiplicative factor of $\sim\!1.3$.

However, this result only applies under the assumption that $\Lambda_0 = 
\Lambda_1 = \Lambda_2$.
If hot Jupiters are less common around lower mass stars, it would be more 
sensible to consider $\Lambda_2<\Lambda_0$, while letting single stars and 
primaries host planets at the same rate.
Therefore in Fig.~\ref{fig:HJ_correction_inputrate_vs_HJratevalues} we let 
$\Lambda_2$ vary, and show the resulting inferred and true hot Jupiter 
($r>8r_\oplus$) rates.
The result is that the inferred rate is nearly independent of 
$\Lambda_2$~--~this is because most ($<1/10$) secondaries are not searchable, 
and so their completeness fraction is much smaller than that of primaries or 
single stars.
This means far fewer detected hot Jupiters orbit secondaries, and so they 
hardly affect the inferred rate.
While the ``true rate'' across the entire population is highly dependent on 
$\Lambda_2$, as one would expect from Eq.~\ref{eq:occ_rate}, the rate around 
singles and primaries (green line) is independent of that around secondaries.
Assuming that RV surveys are measuring the true rate around single stars (or 
primaries), this suggests that binarity might contributes to the 
HJ rate discrepancy at the $\sim 0.2\%$ level, independent of the HJ rate 
around secondaries.

\paragraph{The Rate of Earth Analogs}
The rate (density) of Earth-like planets orbiting Sun-like stars has 
been independently measured by Youdin, Petigura, Dong \& Zhu, 
Foreman-Mackey et al., and Burke et al., (2011, 2013, 2013, 2014, and 2015 
respectively).
These efforts have found that the one-year terrestrial planet occurrence rate 
varies between $\approx 0.03$ and $\approx 1$ per Sun-like star, depending on 
assumptions that are made when retrieving the rate (Burke et al. 2015's 
Fig.~17).

Our model does not explicitly include the rate density's period-dependence, 
because stellar binarity does not appreciably bias the period-dependence of 
occurrence rates measured by transit surveys\footnote{
    Note that stellar binarity does bias the {\it intrinsic} planet 
    occurrence as a function of planetary and binary periods. This is expected 
    from dynamical stability limits in $\geq$3 body systems, and has been 
    observationally confirmed (theory by Holman \& Wiegert 1999, and others 
    including Gongjie; confirmation from Wang et al 2014a, 2014b, Kraus et al 
    2014). 
    However our statement is that inferred rates as a function of planet     
    period should be negligibly affected by this, given the geometric bias 
    against long-period transit detections, and the fact that the period 
    distribution of solar-type binaries peaks at $\approx 100\,{\rm years}$ 
    (Raghavan et al 2010, Fig.~13).
}.
Instead, it allows us to evaluate the difference in the apparent and true rate 
as a function of radius 
(Fig.~\ref{fig:errcases_model_3_log}).
At Earth's radius, the result is that the inferred rate is $0.84\times$ the 
true rate around single stars, assuming that the $\Lambda_i$'s are equal.
Similar to the above case of the hot Jupiters, if we vary the true $\Lambda_2$ 
while keeping $\Lambda_0 = \Lambda_1$, the ratio of the inferred to true rate 
around single stars hardly 
changes; $(\Lambda_{\rm inferred}/\Lambda_1)_{r=r_\oplus} \approx 0.135$, to 
within a few percent, independent of $\Lambda_2$ (see 
Fig.~\ref{fig:earth_inputrate_vs_etaearthratevalues}).
The ratio of the inferred to the true rate, $(\Lambda_{\rm 
inferred}/\Lambda)_{r=r_\oplus}$ varies substantially, but by at most $50\%$ 
in the (unrealistic) limiting case that secondaries do not host planets.


