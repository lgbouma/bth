\section{Conclusion}
\label{sec:conclusion}

This study presented three simple models for the effects of binarity on 
occurrence rates measured by transit surveys.
The most realistic of these models (Model \#3) suggests that binarity leads to 
underestimates of occurrence rates at less than $30\%$ relative 
error.
The model indicates that hot Jupiter rates measured by transit surveys are 
biased to infer $\approx 2$ fewer hot Jupiters per thousand single stars than 
surveys that only measure occurrence rates about single stars ({\it i.e..,} 
radial velocity surveys).
Our third model also indicates that binarity's effects on the measured 
occurrence rates of Earth-sized planets are far smaller than current 
systematic uncertainties.
Though our models are simplistic, their agreement with Barclay \& 
Collaborator's recent detailed simulations indicate that they capture the 
essential ingredients.
