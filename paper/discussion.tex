\section{Discussion}


\paragraph{Does a detected planet orbit the primary or secondary?}
Ciardi et al. (2015) studied the effects of stellar multiplicity on the 
planet radii derived from transit surveys.
They modeled the problem for {\it Kepler}\ objects of interest by matching a 
population of binary and tertiary companions to KOI stars, 
under the assumption that the KIC-listed stars were the primaries.
They then computed planet radius correction factors assuming that {\it 
Kepler}-detected planets orbited the primary or companion stars
with equal probability (their Sec. 5).
Under these assumptions, they found that any given planet's radius is on 
average underestimated by a multiplicative factor of 1.5.

Our models show that the uniform prior on planet assignment overstates 
binarity's population-level effects.
A planet orbiting the secondary does lead to extreme corrections, but these 
cases are rare outliers, because the searchable volume for secondaries is so 
much smaller than that for primaries.
Phrased in terms of the completeness, in our Model \#3 only $\sim 6\%$ of 
selected secondaries are searchable, compared to $\sim 60\%$ of selected 
primaries.
This means that when high-resolution imaging discovers a binary companion in 
system that hosts a detected transiting planet, the planet is much more likely 
to orbit the primary.
It also means that the average radius correction factor across the entire {\it 
Kepler} sample is smaller than that advertised by Ciardi et al.

\paragraph{On the utility for future occurrence rate measurements}
Though they will be difficult to distinguish from false positives, {\it TESS}\ 
is expected to discover over $10^4$ giant planets (Sullivan et al. 2015).
One possible use of this overwhelmingly large sample will to measure an
occurrence rate of short-period giant planets.
Our work indicates that if this measurement is to be more precise than $\sim 
30\%$, binarity cannot be neglected.

\paragraph{What about $\eta_\oplus$?}

\paragraph{What about the radius gap?}

\paragraph{What about {\it Kepler}?}