\section{Introduction}

An astronomer who does not believe in stellar multiplicity wants to measure the
mean number of planets of a certain type per star of a certain type.
They perform a signal-to-noise limited transit survey and detect $N_{\rm det}$ 
transit signals that appear to be from planets of the desired type.  They 
calculate the number of stars $N_\star$ that seem to be of the desired type, 
and searchable for the planet.  ``Searchable stars'' are the stars for which 
planets are observed with 100\% detection efficiency. 
Correcting for the geometric 
transit probability $f_{\rm g}$, they compute an apparent occurrence rate 
$\Lambda_{\rm a}$:

\begin{equation}
\Lambda_{\rm a} = \frac{N_{\rm det}}{N_\star} \times \frac{1}{f_g}.
\end{equation}

There are many potential pitfalls.  Some genuine transit signals can be missed
by the detection pipeline.  Some apparent transit signals are spurious, from
noise fluctuations, failures of `detrending', or instrumental effects.  Stars
and planets can be misclassified due to statistical and systematic errors in
the measurements of their properties.  Poor angular resolution causes false
positives due to blends with background eclipsing binaries. {\it Et cetera}.

Here we focus on problems that arise from the fact that stars of the desired
type often exist in multiple star systems.
For simplicity, we only consider binaries, and we assume that they are all 
spatially unresolved.

An immediate complication is that~--~due to dynamical stability or some 
aspect of planet formation~--~the occurrence rate of planets may differ 
between binary and single-star systems.
If ``occurrence rate'' is defined purely as the mean number of planets within 
set radius and period bounds per star in a given interval of say $\Delta 
M_\star$, it must implicitly marginalize over stellar multiplicity.
Otherwise, one must discuss ``occurrence rates in single star systems'', 
``occurrence rates about primaries'', and
``occurrence rates about secondaries'' (this has been noted by Wang et al., 
2015)

Outside of astrophysical differences, there are observational biases.
A given apparently-searchable star may truly be a single star of the desired 
type. If not one of the following must be true:
\begin{enumerate}	
    \item The apparently-searchable star is a binary system, with two 
    searchable 
    stars of the desired type. $N_\star$ is under-counted by one for every 
    such 
    system.
    %
    \item The apparently-searchable star is a binary system, with one 
    searchable star of the desired type. 
    $N_\star$ is correctly counted for every such system.
    There may be systematic errors in stellar parameter estimates of such 
    systems, but in this work we neglect these.
    %
    \item  The apparently-searchable star is a binary system, in which neither 
    of the components is a searchable star, but at least one is of the desired 
    type. 
    $N_\star$ is over-counted by one for every such system.
\end{enumerate}

One might also imagine an apparently-searchable star which in no stellar 
component is of the desired type.
We will not consider errors of this type.  We will assume that all the 
apparently-searchable stars are either single stars of
the desired type, or binaries in which either the primary or else both 
components are of the desired type.

When counting $N_{\rm det}$, the detected signals from planets of the desired 
type, neglecting binarity introduces the following error cases:
\begin{enumerate}
    \item 
    A detected signal is from a star of undesired type (by the above 
    assumptions, 
    the secondary of a binary system), from a planet of undesired type, and is 
    incorrectly counted as a planet of desired type.
    %
    \item
    A detected signal is from a star of desired type (the primary or secondary 
    of 
    a binary), from a planet of undesired type, and is incorrectly counted 
    as a planet of desired type.
    This and the case above occur when the host star parameters are 
    miscalculated 
    ({\it e.g.,} the host star is the faint secondary, but is assumed to be 
    primary), or when the constant diluting light from the binary companion is 
    not corrected.
    $N_{\rm det}$ should be lowered by 1 for every such case.
    %
    \item
    A detected signal is from a star of desired type, from a planet of desired 
    type, but incorrectly counted as planet of undesired type.
    $N_{\rm det}$ should be raised by 1 for every such case.
    \item
    An {\it undetected} signal from a planet of the desired type, around a 
    star of 
    the desired type, was not detected because of dilution from the companion 
    (because the SNR floor of the survey is set for single stars). 
    This happens during case \#3 in $N_\star$-counting errors (the 
    apparently-searchable star is a binary system, in which neither component 
    is 
    in fact searchable). If $N_\star$ is corrected for this, {\it i.e.,} the 
    system is correctly counted as ``not searchable'', no correction to 
    $N_{\rm 
        det}$ is needed.
\end{enumerate}

Errors (1)-(3) in calculating $N_{\rm det}$ can be broadly described as 
``radius misclassification'', while error (4) is an error in the assumed 
survey completeness.



\paragraph{The Hot Jupiter Rate Discrepancy}

Radial velocity surveys of nearby bright Sun-like stars have reported hot 
Jupiter ($a<0.1\,{\rm AU}, P\lesssim 10\,{\rm day}$) occurrence rates of
$12\pm 1$, $15\pm 6$, $8.9 \pm 3.6$, and $12.0 \pm 3.8$ per thousand stars
(Marcy et al 2005, Cumming et al 2008, Mayor et al 2011, Wright et al 2012 
respectively).

In transit surveys, OGLE-III reported an absolute rate of $3.1^{+ 
4.3}_{-1.8}$ hot Jupiters with $P<5\,{\rm days}$ per thousand stars (Gould et 
al 2006).
The SuperLupus survey quoted $10^{+27}_{-8}$ per thousand stars, allowing 
planets with periods less than 10 days (Bayliss \& Sackett 2011).
Through the {\it Kepler}\ survey, Howard et al. (2012) later reported a rate 
of 
$4 \pm 1$ per thousand Sun-like stars.
This rate was for $P<10\,{\rm day}, 8-32r_\oplus$ planets, and 
was specific to their ``solar subset''\footnote{The ``solar subset'' was 
defined as $4100<T_{\rm eff}/{\rm K}<6100$, {\it Kepler}\ 
magnitude $<15$, $4.0 < \log g < 4.9$, and only took planets with measured 
signal to noise $>10$.
}.
Expanding their sample to fainter magnitudes, they quoted a rate of $5 \pm 
1$.
Expanding down to $r_p>5.6r_\oplus$, to avoid excluding hot Jupiters reported 
by RV surveys, they reported $7.6 \pm 1.3$.
Recent work by Petigura et al. using the updated parameters of the 
California-{\it Kepler}\ Survey has found a rate of $X.X \pm 
Y.Y$ (2017, in preparation)

The trend is that hot Jupiter occurrence rates measured by transit 
surveys seem to be marginally lower than those found by radial velocity 
surveys.
While the actual discrepancy is of sub-$3\sigma$ significance,
one reason to expect a difference in the rates is that the populations have 
different metallicities.
As originally argued by Gould et al. (2006), the RV sample is biased towards 
metal-rich stars, which have been measured by RV surveys to preferentially 
host more giant planets (Santos et al 2004, Fischer and Valenti 2005).
The {\it Kepler}\ sample specifically has been measured to be more metal poor 
than the local neighborhood, with a mean metallicity of $[{\rm M/H}]_{\rm 
mean}\approx -0.05$ (Dong et al., 2014; Guo et al., 2017).
Studying the problem in detail, Guo et al. recently argued that the 
metallicity difference could account for a $\approx 0.1\%$ difference in the 
measured rates between the CKS and {\it Kepler}\ samples~--~not a $\approx 
0.5\%$ difference.
These authors concluded that ``other factors, such as binary contamination and 
imperfect stellar properties'' must also be at play (Guo et al., 2017).

Aside from surveying stars of different metallicities, radial velocity and 
transit surveys differ in how they treat binarity.
Radial velocity surveys typically reject both visual and spectroscopic binaries
({\it e.g.}, Wright et al. 2012).
Transit surveys typically observe binaries, but the question of whether they 
were searchable to begin with is usually left for later interpretation.
In spectroscopic follow-up of transiting candidates, the prevalence of 
astrophysical false-positives may also lead to a bias against confirmation of 
transiting planets in binary systems.

Ignoring these complications, in this work we focus on whether
binarity might intrinsically bias occurrence rate measurements, simply 
through its effects on the number of searchable stars and the apparent radii 
of detected planets.

To progressively gain intuition, we study the following idealized transit 
surveys:
\begin{itemize}
    \item Model \#1: fixed stars, fixed planets, twin binaries;
    \item Model \#2: fixed planets and primaries, varying secondaries;
    \item Model \#3: fixed primaries, varying planets and secondaries.
\end{itemize}
In Sec.~\ref{sec:numerical_methods}, we introduce our numerical approach 
to the problem, and in Sec.~\ref{sec:analytic_preliminaries} we clarify our 
terminology.
We present the analytic and numerical results in 
Secs.~\ref{sec:model_1}-\ref{sec:model_3}, where each subsection corresponds 
to each model above.
We interpret these calculations throughout, and in 
Sec.~\ref{sec:discussion} discuss their relevance to various questions in 
the interpretation of transit survey occurrence rates. 