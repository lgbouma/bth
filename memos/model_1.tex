\section*{Generalities}

Define the occurrence rate density, $\Gamma$,
as the expected number of planets per star per natural logarithmic bin 
of planetary and stellar phase space:
\begin{equation}
\Gamma(\vec{x}) \equiv \frac{{\rm d}^n(N_p/N_\star)}{\prod_{i=1}^{n} {\rm d} 
\ln x_i}
= \frac{d^n\Lambda}{\prod_{i=1}^{n} {\rm d} \ln x_i  }.
\end{equation}
$\vec{x}$ is an $n$-dimensional list of the continuous physical properties 
that might affect the occurrence rate density. For example,
$\vec{x}=(r,P,R)$ where $r$ is the planet radius, $P$ is its orbital period, 
and $R$ is the host star radius.
In the latter equality, $\Lambda$ is the occurrence rate~--~a quantity found 
by integrating the rate density over a specified volume of phase space ({\it 
e.g.}, Foreman-Mackey et al, 2014).

According to the previous definition, the rate density implicitly marginalizes 
over stellar multiplicity.
For simplicity, this work only considers single and binary star systems.
Then for a selected population of stars and planets the rate density can be 
written
\begin{equation}
\Gamma(\vec{x}) = \sum_{i=0}^{2} w_i \Lambda_i p_i(\vec{x}) \equiv \sum_i 
\Gamma_i(\vec{x}),
\end{equation}
where $i=0$ corresponds to single star systems, $i=1$ primaries of a binaries, 
and $i=2$ secondaries of binaries.
$\Lambda_i$ is the occurrence rate for the $i^{\rm th}$ system 
type, $p_i(\vec{x})$ is the corresponding joint probability density function, 
and the weights are given by 
\begin{equation}
w_i = N_i/N_{\rm tot},
\end{equation}
for $N_{\rm tot} = \sum_i N_i$ the total number of selected stars, and 
$N_0,N_1,N_2$ the number of selected single stars, primaries, and 
secondaries respectively.

A transit survey will have a rate density of detected planets $\hat{\Gamma}$, 
which will be the (dot) product of the rate density and the detection 
efficiency $Q(\vec{x})$:
\begin{equation}
\hat{\Gamma}(\vec{x}) = \sum_i Q_i(\vec{x}) \Gamma_i(\vec{x}) 
\equiv \sum_i \hat{\Gamma}_i(\vec{x}),
\label{eq:detected_rate_density}
\end{equation}
where again the index $i$ is over each type of system (singles, primaries, 
and secondaries).
The detection efficiency includes the geometric transit probability, as well 
as any incompleteness effects.




\section{Model \#1: fixed stars, fixed planets, twin binaries}
\label{sec:model_1}

Consider a universe in which all planets are identical, and all stars are 
either single or twin stars with otherwise identical physical properties.
Then $\vec{x} = (r,R,a)$, and 
\begin{equation}
p_i(\vec{x})= \delta(r-r_p)\delta(R-R_\star)\delta(a-a_p)
\equiv \delta^3(r_p,R_\star,a_p),
\label{eq:model_1_prob}
\end{equation}
for $r_p$ and $R_\star$ some fixed planet and stellar radii, $a_p$ a fixed 
semi-major axis, and $\delta$ the Dirac delta function, whose latter compact 
form will be used for brevity.

The occurrence rate density for this model is
\begin{equation}
\Gamma(r,R,a) = \sum_i w_i \Lambda_i \delta^3(r_p,R_\star,a_p),
\label{eq:model1_occ_rate_density}
\end{equation}
and the occurrence rate over any interval that includes $r_p$, $R_\star$, and 
$a_p$ is
\begin{equation}
\Lambda = \sum_i w_i \Lambda_i = \frac{ \sum_i N_i \Lambda_i }{N_{\rm tot}}.
\end{equation}
The rate is zero over intervals that do not.

To express the rate density of detected planets, $\hat{\Gamma} = \sum 
Q_i\Gamma_i$, we need the detection efficiencies for each system type, which 
are products of the geometric and selection probabilities:
\begin{align}
Q_i(\vec{x}) &= Q_{g,i}(\vec{x}) Q_{c,i}(\vec{x}),\quad {\rm where}\ 
\vec{x}=(r,R,a).
\label{eq:general_detection_efficiency}
\end{align}
Similar to Pepper et al. (2003), but in a new context, we take $Q_c$ as the 
ratio of the number of stars that were searchable to the number of stars that 
were selected.
Assuming a homogeneous distribution of stars, this gives
\begin{equation}
Q_{c,i}(\vec{x}) = \left(
\frac{d_{{\rm det},i}(\vec{x})}{d_{\rm sel}(\vec{x})}
\right)^3,
\end{equation}
for $d_{\rm sel}$ the maximum distance to which surveyed stars are selected, 
and $d_{{\rm det},i}$ the maximum distance to which planets can actually be 
detected about the $i^{\rm th}$ system type.
Note that $d_{\rm sel} \geq d_{{\rm det},i}$.
In a signal-to-noise limited transit survey in which the observer does not 
know which stars are binaries, 
\begin{equation}
d_{\rm sel} \propto (r/R)^2 (L_{\rm sys} T_{\rm dur} A N_{\rm tra})^{1/2},
\end{equation}
for $L_{\rm sys}=L_1(1+\gamma_R)$ the system luminosity, $T_{\rm dur}$ the 
transit duration, $A$ the detector area, and $N_{\rm tra}$ 
the number of observed transits.
However,
\begin{equation}
d_{{\rm det},i} \propto \mathcal{D}_i(r/R)^2 (L_{\rm sys} T_{\rm dur} A N_{\rm 
tra})^{1/2},
\label{eq:d_det_i}
\end{equation}
for the dilution $\mathcal{D}_i$ given by
\begin{align}
\mathcal{D}_i
&=
\left.
\begin{cases}
1 & \text{for } i=0,\ {\rm single} \\
L_1 / L_d = (1+\gamma_R)^{-1}, & \text{for } i=1,\ {\rm primary} \\
\gamma_R L_1 / L_d = (1 + \gamma_R^{-1})^{-1}, & \text{for } i=2,\ {\rm 
secondary},
\end{cases}
\right.
\label{eq:dilution}
\end{align}
where the light ratio $\gamma_R$ of a given binary is defined as the ratio of 
the luminosity of the secondary to the primary.

The maximum detectable distance to single stars is assumed to be known, and so 
$d_{{\rm sel},0} = d_{{\rm det},0}$.
For binary systems there is a necessary incompleteness, and combining 
Eqs.~\ref{eq:general_detection_efficiency} through~\ref{eq:dilution} in the 
case of twin binaries yields
\begin{align}
Q_i(\vec{x})
&=
\left.
\begin{cases}
R/a, & \text{for } i=0\ {\rm single} \\
(R/a) (1+\gamma_R)^{-3} = (R/8a), & \text{for } i=1\ {\rm primary} \\
(R/a) (1+\gamma_R^{-1})^{-3}\gamma_R^{-5/\alpha} = (R/8a), & \text{for } i=2\ 
{\rm 
secondary}.
\end{cases}
\right.
\label{eq:model1_detection_efficiency}
\end{align}

The factor of $\gamma_R^{-5/\alpha}$ in the latter expression of 
Eq.~\ref{eq:model1_occ_rate_density} comes from the mass-luminosity-radius 
relation of stars in the model: we assume $R\propto M \propto L^{1/\alpha}$.
For $\gamma_R=1$ this term is unimportant, but it will later become relevant.

Summarizing, since we have written the rate density for each system type
(Eq.~\ref{eq:model1_occ_rate_density}) and the detection efficiency for each 
system type (Eq.~\ref{eq:model1_detection_efficiency}), we have fully 
specified the rate density of detected planets, $\hat{\Gamma}=\sum Q_i 
\Gamma_i$, in addition to the true rate density.


\subsection{What does an observer ignoring binarity infer? }

Consider an observer who ignores binarity.
They assume a detection efficiency $\tilde{Q}=Q_0$,
measure a detected planet rate density $\tilde{\Gamma}$, 
and infer an apparent rate density $\Gamma_a$.
Analogous to Eq.~\ref{eq:detected_rate_density},
\begin{equation}
\tilde{\Gamma} = \Gamma_a \tilde{Q}.
\end{equation}
Accounting for dilution, one can show
\begin{equation}
\Gamma_a = 
w_a \Lambda_0 \delta^3(r_p, R_\star, a_p) +
w_b (\Lambda_1 Q_{c,1} + \Lambda_2 Q_{c,2}) 
				\delta^3(r_p/\sqrt{2}, R_\star, a_p),
\end{equation}
for $w_a = N_0/(N_0+N_1)$, and $w_b = N_1/(N_0+N_1)$.
This observer miscounts the number of total searched stars, does not correct 
for completeness, and misclassifies the planetary radii because of dilution.

\subsection{Correction to inferred rate density and inferred rate}

Define the rate density correction factor, $X_\Gamma$, as the ratio of the 
apparent to true rate densities:
\begin{equation}
X_\Gamma \equiv \frac{\Gamma_a}{\Gamma}.
\end{equation}
This factor can be a function of whatever parameters $\Gamma_a$ and $\Gamma$ 
depend on; in this study, the planet radius is most relevant.
For the twin-binaries model,
\begin{equation}
X_\Gamma(r)
=
\frac{w_a \Lambda_0\delta^3(r_p) + 
	w_b(\Lambda_1 Q_{c,1} + \Lambda_2 Q_{c,2}) \delta^3(r_p/\sqrt{2})  }
	{(w_0\Lambda_0 + w_1\Lambda_1 + w_2\Lambda_2)\delta^3(r_p)}
	\label{eq:model1_correction}
	\end{equation}
where $\delta^3(r_p)$ is shorthand for $\delta^3(r-r_p,R-R_\star,a-a_p)$.

If we take the rates $\Lambda_i$ to be equal, applying the definitions of 
the weights gives a rate density correction factor at $r=r_p$ of
$X_\Gamma(r_p) = (1+\mu)^{-1}$, where 
\begin{align}
\mu \equiv \frac{N_1}{N_0} &=
\frac{n_b}{n_s} \left(\frac{d_{\rm sel,b}}{d_{\rm sel,s}}\right)^3 = 
\frac{{\rm BF}}{1-{\rm BF}} (1+\gamma_R)^{3/2},
\label{eq:mu_definition}
\end{align}
for $n_b$ and $n_s$ the number density of binaries and singles in a 
volume limited sample.
Using Raghavan et al. (2010)'s $0.7-1.3M_\odot$ multiplicity fraction as our 
binary fraction, we set ${\rm BF}=0.44$.
The resulting correction to the rate density is $X_\Gamma(r_p) \approx 0.31$. 
The correction at $r_p/\sqrt{2}$ is infinite.
%beta = 2.2223355980148636


If we assume that $\Lambda_0 = \Lambda_1$, but that $\Lambda_2=0$, we find 
$X_\Gamma(r_p) = (1+2\mu)/(1+\mu)^2$.
Taking the same binary fraction, this evaluates to $X_\Gamma(r_p)\approx 0.53$.

In passing, note that a correction to the inferred rate, $X_\Lambda$, can be 
defined analogously:
\begin{equation}
X_\Lambda \equiv \frac{\Lambda_a}{\Lambda}.
\end{equation}
For the twin binary model, the correction to the rate is the same as that to 
the rate density.