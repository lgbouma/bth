%\documentclass[12pt,preprint]{aastex}
%\documentclass[preprint2,12pt]{aastex}
\documentclass{emulateapj}
%\usepackage{apjfonts}
\usepackage{graphicx}
\usepackage{amssymb}
\usepackage{amsmath}

\shortauthors{Name et al.}
\shorttitle{Best transiting planets}

\begin{document}

% ------------------------------------------------------------------------
% New commands
%
\def\ltsima{$\; \buildrel < \over \sim \;$}
\def\lsim{\lower.5ex\hbox{\ltsima}}
\def\gtsima{$\; \buildrel > \over \sim \;$}
\def\gsim{\lower.5ex\hbox{\gtsima}}
\def\tess{{\it TESS} }
\def \teff {T_{\rm eff}}
\def \phir {\Phi_{\rm R}}
\def \fov {24$^{\circ}$}
\def \pixsz {21.1''}
\def \aeff {69.1 cm$^2$ }    
\def \epd {105 mm}                          
                                                                                          
% -------------------------------------------------------------------------
%

\bibliographystyle{apj}

\title{ A Toy Analytic Transit Survey }

\author{
  LGB, KM, JNW
}

% \journalinfo{Draft version}
\slugcomment{Memo for internal use}

%\altaffiltext{1}{Princeton University}

\begin{abstract}

We derive the signal to noise distribution of detected planets in a toy transit 
survey. We give corresponding expressions for the number of detected planets. 
We then discuss the effect of various errors that one makes by ignoring 
binarity when deriving occurrence rates.

\end{abstract}

\keywords{planets and satellites:\ detection}

\section*{Statement of problem}

Ah, transiting planets! We learn so much by studying them. But how much can we 
actually learn, and how much is messed up by binarity?

Imagine the following transit survey:
\begin{itemize}
\item Take a magnitude-limited sample of ``stars'' of a single spectral class. 
For instance, G2V dwarfs.  In other words, choose all ``points'' on-sky with 
$m_\lambda < m_{\lambda,{\rm lim}}$, or equivalently with $F_\lambda > 
F_{\lambda,{\rm lim}}$, and also with colors that make you think they are this 
spectral class.
\item The true population of ``points'' (hereafter, systems) contains both 
single and double star systems. Single star systems have luminosity in the 
observed bandpass $L_1$, radii $R_1$, and effective temperature $T_{\rm eff,1}$.
Double star systems have luminosity in the observed bandpass $L_d = 
(1+\gamma_R)L_1$, for $\gamma_R = L_2/L_1$ the ratio of the luminosity of the 
secondary to the primary. The ratio of the two number densities in a 
volume-limited sample is the binary fraction\footnote{The binary fraction is 
equivalent to the multiplicity fraction if there are no triple, quadruple, 
$\ldots$ systems.}.
\item The true population of planets around these stars is as follows:
	\subitem A fraction $\Gamma_{t,s}$ of stars in single star systems 
	have a planet of radius $R_p$, with orbital period $P$.
	\subitem A fraction $\Gamma_{t,d}$ of each star in a double star 
	system has a planet of radius $R_p$, with orbital period $P$. For instance, 
	if $\Gamma_{t,s} = \Gamma_{t,d} = 0.1$, on average each double 
	system contributes 0.2 planets, and each single system 0.1 planets.
    Any astrophysical difference in planet formation between singles and 
    binaries is captured by these two terms.
\end{itemize}

A signal to noise limited transit survey is then conducted. All planets with 
${\rm S/N} > {\rm (S/N)_{min}}$ are detected. The rest are not.
Note that while the S/N threshold does imply a minimum stellar flux about which 
planets of a given radius and period can be detected, this minimum flux is 
generally different from the minimum used to define the original magnitude cut 
for a sample [see Batalha et al, 2010].

Consider the following set of questions:
\begin{enumerate}
\item How many single and double star systems, respectively, are in the sample?
Correspondingly, how many stars are in the sample?

\item How many planets are in the sample? (Orbiting single stars, and orbiting 
double stars respectively).

\item What is the true occurrence rate?

\item How many planets are detected?

\item What occurrence rate does astronomer A, who has never heard of binary 
star systems, derive for planets of radius $R_p$ and period $P$?

\item What occurrence rate does astronomer B, who accounts for the ``2 for 1'' 
effect of binarity (\textit{i.e.} that the sample actually has more stars than 
astronomer A thought) derive?

\item What about astronomer C, who accounts for ``2 for 1'' \textit{and} 
misclassification due to diluted radii? In other words, astronomer C has lots 
of Keck time, and did high resolution imaging followup on every candidate, and 
correctly classifies the planetary radii.

\item What about astronomer D, who additionally notes the importance of 
completeness?
\end{enumerate}


\section{How many stars are in the sample?}

Let $N_s$ be the number of single star systems, and $N_d$ the number of double 
star systems. Then the total number of stars in the sample can be written
\begin{equation}
N_{\rm stars} = N_s + 2N_d.
\end{equation}

In a magnitude-limited sample in which stars are uniformly distributed in 
volume, the number of stars will be the number density times the volume.
If the volume is taken to be a sphere over which the number density is uniform,
\begin{equation}
N_i = n_{i} \frac{4\pi}{3} d_{{\rm max},i}^3,
\label{eq:number_systems}
\end{equation}
for $i\in{\rm \{single,double\} } \equiv { \{s,d\} }$, and
\begin{equation}
\frac{n_{d}}{n_{s}} = {\rm binary\ fraction} \equiv {\rm BF}
\end{equation}
by definition. The absolute normalization of the number density is a measured 
quantity, as is the binary fraction. For G2V dwarfs, the latter is $\approx 
0.45$ [Duchene \& Kraus, 2013]. The former is given by [Bovy 2017].

$d_{{\rm max},i}$ in Eq.~\ref{eq:number_systems} is the maximum distance 
corresponding to the given magnitude limit:
\begin{equation}
d_{{\rm max},i} = \left(\frac{L_{i}}{4 \pi F_{{\rm lim}}}\right)^{1/2},
\label{eq:d_max}
\end{equation}
where the limiting flux in the bandpass $F_{{\rm lim}}$ can also be 
stated in terms of the limiting magnitude $m_{{\rm lim}}$,
\begin{equation}
m_{{\rm lim}} = m_{0} - \frac{5}{2} \log_{10} \left(\frac{F_{{\rm 
lim}}}{F_{0}}\right),
\end{equation}
for $m_{0}$ a zero-point magnitude and $F_{0}$ its corresponding flux (as 
always, everything is implicitly written in a defined bandpass).

In Eq.~\ref{eq:d_max}, again $i\in{\rm \{single,double\} }$, and as a 
consequence the maximum distance to which binary stars will be selected is 
greater than that of single stars, simply as a consequence of imposing a 
magnitude cut.
The ratio of double to single systems is
\begin{align}
\frac{N_d}{N_s} &= 
	\frac{n_d}{n_s} \left(\frac{d_{{\rm max}, d} }{d_{{\rm max}, s}}\right)^3 \\
&= {\rm BF} \times (1+\gamma_R)^{3/2}.
\end{align}
In the nominal case of twin binaries ($\gamma_R = 1$), with a binary fraction 
${\rm BF} = 0.5$, there are 
$\sqrt{2}$ more binary systems than single systems in the sample.
Correspondingly, there are $2\sqrt{2}$ more stars in binary systems than stars 
in single systems.

As a comment on Eq.~\ref{eq:number_systems}, if 
we wished to write a stellar number density profile that accounted for the 
vertical structure of the Milky Way, we might choose a profile either 
$\propto \exp(-z/H)$, or $\propto {\rm sech}^2(z/H)$ for $z$ the distance from 
the galactic midplane and $H$ a scale-height. Both density profiles would lead 
closed form analytic solutions.



\section{How many planets are in the sample?}

The number of planets in the sample is
\begin{align}
N_{\rm planets} =& N_{\rm planets\ in\ single\ star\ systems}  +  \\
				  &\quad\quad N_{\rm planets\ in\ double\ star\ systems} 
				  \nonumber \\
			   =& \Gamma_{t,s} N_s + 2 \Gamma_{t,d} N_d.
\end{align}

The factor of 2 accounts for the fact that there are twice as many stars in 
double star systems.



\section{What is the true occurrence rate?}
\label{sec:true_rate}

The ``true occurrence rate'' is the average number of planets per star. Thus

\begin{align}
\Gamma_t &= \frac{N_{\rm planets}}{N_{\rm stars}} \\
\Gamma_t &= \frac{\Gamma_{t,s} N_s + 2 \Gamma_{t,d} N_d}{N_s + 2N_d}.
\label{eq:true_occ}
\end{align}



\section{How many planets are detected?}
The total number of planet detections is the sum of the number of planets 
detected in single star systems $N_{{\rm det},s}$ and the number of planets 
detected in double star systems $N_{{\rm det},d}$.
These can be expressed individually. 
The former is
\begin{equation}
N_{{\rm det},s} = N_s \Gamma_{t,s} f_{s,{\rm S/N} > {\rm (S/N)_{min}}},
\label{eq:N_det_s}
\end{equation}
where the product $N_s \Gamma_{t,s}$ is the number of planets in the single 
star systems of the sample, and $f_{s,({\rm S/N} > {\rm (S/N)_{min}})}$ is the 
fraction of these planets that have signal to noise greater than the minimum 
detection threshold.
\begin{equation}
N_{{\rm det},d} = 2 N_d \Gamma_{t,d} f_{d,{\rm S/N} > {\rm (S/N)_{min}}},
\label{eq:N_det_d}
\end{equation}
where now $2 N_d \Gamma_{t,d}$ is the number of planets in the double star 
systems of the sample, and the completeness term must account for any 
differences in the signal to noise distribution that come from stellar 
binarity.

As a note of warning, Eqs.~\ref{eq:N_det_s} and~\ref{eq:N_det_d} ignore the 
geometric transit probability, $\approx R_\star / a$. This means we are 
assuming that every existing planet in this universe transits. This assumption 
can easily be rectified, if desired.

\subsection{Analytic completeness}
The only terms we have yet to compute are the completeness terms, $f_{i,{\rm 
S/N} > {\rm (S/N)_{min}}}$ for $i\in{\rm \{single,double\} }$. 
We proceed as follows.

The signal ${\rm S}$ for a box-car train transiting planet is
\begin{align}
{\rm S} &= \delta \mathcal{D} \\
&= \left(\frac{R_p}{R_\star}\right)^2 \mathcal{D},
\end{align}
for $R_p$ the planet's radius, $R_\star$ that of its host star, and 
$\mathcal{D}$ the dilution parameter defined as
\begin{equation}
  \mathcal{D} = 
  \Bigg\{\begin{array}{lr}
  L_1 / L_d, & {\rm\ if\ binary\ and\ target\ primary}\\
  \gamma_R L_1 / L_d, & {\rm\ if\ binary\ and\ target\ 
  secondary}\\
  1, & {\rm\ if\ single},
  \end{array}
  \label{eq:dilution}
\end{equation}
where $L_1, L_d,$ and $\gamma_R$ were defined in the opening monograph.

Assuming the only source of noise is Poissonian counting noise, the noise ${\rm 
N}$ can be written
\begin{equation}
{\rm N} = \frac{1}{\sqrt{N_\gamma}},
\end{equation}
for $N_\gamma$ the number of photons received by the detector. This noise model 
is a useful simplification -- see [Howell 2006, pg 75] for the full 
CCD equation.
The number of received photons can be written
\begin{equation}
N_\gamma = F^{\rm N}_\gamma A N_{\rm tra} T_{\rm dur},
\end{equation}
for $F^{\rm N}_\gamma$ the photon number flux from the system [${\rm 
ph\,cm^{-2}\,s^{-1}}$], 
$A$ the detector area, $T_{\rm dur}$ the transit duration, and $N_{\rm tra} $ 
the number of transits observed, which is multiplied in assuming the transits 
are ``phase-folded''.
	
Thus the signal to noise ratio can be written
\begin{equation}
{\rm S/N} = \delta \mathcal{D} \sqrt{F^{\rm N}_\gamma A N_{\rm tra} T_{\rm 
dur}}.
\label{eq:snr_ivory_tower}
\end{equation}

In passing, given the parameters that define a survey and planet type, 
Eq.~\ref{eq:snr_ivory_tower} would need to be re-expressed with 
$N_{\rm tra}$ roughly the ratio of the observing baseline to the planet period, 
and $T_{\rm dur}$ a function of $R_\star, P, a$, and impact parameter $b$, and 
then perhaps averaged over $b$. We leave them as-is for subsequent 
development.

The interesting term in Eq.~\ref{eq:snr_ivory_tower} that changes between stars 
of the same binarity class in our idealized sample is the square root of 
$F^{\rm N}_\gamma$.
This is the term that leads to a distribution of signal to noises for different 
stars.
The completenesses $f_{i,{\rm S/N} > {\rm (S/N)_{min}}}$ can be directly 
expressed in terms of those probability density functions:
\begin{equation}
f_{i,{\rm S/N} > {\rm (S/N)_{min}}} = 
	\int_{{\rm (S/N)_{min}}}^{\infty} 
		{\rm d}\left(\frac{{\rm S}}{{\rm N}}\right)_i \ 
		{\rm prob}\left(\frac{{\rm S}}{{\rm N}}\right)_i.
\label{eq:completeness_long}
\end{equation}
We keep the subscript $i$ because the signal to noise distributions are 
different for the cases of single star systems ($i=s$) and double star systems 
($i=d$).
Notably:
\begin{itemize}
	\item The dilution differs (Eq.~\ref{eq:dilution}).
	\item The photon number flux from the system differs.
\end{itemize}

To simplify notation, we let $x_i \equiv {\rm (S/N)}_i$, and rewrite 
Eq.~\ref{eq:completeness_long} as
\begin{equation}
f_{i,x > x_{\rm min}} = 
\int_{{\rm x_{\rm min}}}^{\infty} 
{\rm d}x_i \ 
{\rm prob}(x_i).
\label{eq:completeness_short}
\end{equation}


\subsection{Deriving ${\rm prob}(x_i)$}
We want expressions for the probability density function of the observed signal 
to noise ratio, ${\rm prob}(x_i)$, for both the single and binary 
system case.

First, note that a star placed uniformly in the volume of the search space will 
have a probability density function for its distance $r$ from the origin of
\begin{equation}
{\rm prob} (r) = \frac{3 r^2}{d_{\rm max}^3},
\end{equation}
where the appropriate maximum distances should be substituted per 
Eq.~\ref{eq:d_max}.
Noting the transformation rule for probability density functions, we can 
evaluate the probability of a star having a observed number flux $F^{\rm 
N}_{i,\gamma}$ in the bandpass,
\begin{align}
{\rm prob} (F^{\rm N}_{i,\gamma}) 
&= {\rm prob}(r(F^{\rm N}_{i,\gamma}))
	\left| \frac{{\rm d} r}{{\rm d} F^{\rm N}_{i,\gamma}} \right| \\
&= \frac{3}{2 d_{\rm max}^3} c_i^{3/2} (F^{\rm N}_{i,\gamma})^{-5/2},
\label{eq:pdf_observed_flux}
\end{align}
where in the latter equality we have written a ``number luminosity'' $c_i$ 
(units of inverse time) defined for $i\in{\rm \{single,double\} }$ as
\begin{equation}
c_i = 
\Bigg\{\begin{array}{lr}
R_1^2 F^{\rm N}_{s1,\gamma}, & {\rm\ if\ single}\\
R_1^2 F^{\rm N}_{s1,\gamma} + R_2^2 F^{\rm N}_{s2,\gamma} & {\rm\ if\ double}.
\end{array}
\label{eq:c_i}
\end{equation}
In Eq.~\ref{eq:c_i}, $F^{\rm N}_{s1,\gamma}$ and $F^{\rm N}_{s2,\gamma}$ are 
the photon number fluxes at the surfaces of the stars. To derive 
Eq.~\ref{eq:pdf_observed_flux}, we simply scaled these by the distance:
\begin{equation}
F^{\rm N}_{i,\gamma} = \frac{c_i}{r^2}.
\label{eq:flux_at_detector}
\end{equation}

The surface photon number fluxes $F^{\rm N}_{si,\gamma}$ in Eq.~\ref{eq:c_i} 
are usually evaluated numerically, by convolving the wavelength-specific photon 
flux density of a star with the dimensionless spectral response function of the 
instrument. In other words,
\begin{equation}
F^{\rm N}_{s,\gamma} = \int F_\lambda T_\lambda \, {\rm d}\lambda.
\label{eq:pfd}
\end{equation}
The wavelength-specific photon flux density $F_\lambda$ [${\rm 
ph\,cm^{-2}\,s^{-1}\,\AA^{-1}}$] could be from Pickles' library, or could be 
a blackbody function. The transmission function is, up to factor of order 
unity, a step function between two wavelengths $\lambda_{\rm min}$ and 
$\lambda_{\rm max}$.
If we assume a blackbody source, Eq.~\ref{eq:pfd} becomes
\begin{equation}
F^{\rm N}_{s,\gamma} = 8\pi c \left(\frac{k T}{h c}\right)^3 
\int_{hc/(\lambda_{\rm max} kT)}^{hc/(\lambda_{\rm min} kT)}
\frac{u^2}{e^u - 1} \,{\rm d} u,
\end{equation}
which can be evaluated numerically\footnote{It may help in the numerics to note 
that infinite series representations of this type of dimensionless integral 
exist and converge quickly. For instance, one can show that
\begin{equation}
\int_{0}^{a} \frac{u^3}{e^u -1} \,{\rm d}u = \sum_{n=1}^{\infty} \frac{6}{n^4} 
- \frac{e^{-an}}{n^4} (6 + 6an + 3(an)^2 + (an)^3).
\nonumber
\end{equation}
A similar expression exists for the similar integral in the text. 
[Michels 1968] explains an analogous derivation.
}.

The importance of the functional form of $F^{\rm N}_{s,\gamma}$ is that it is 
to first order only a function of the blackbody temperature and the bandpass 
wavelength bounds.
Thus in the most general case $c_i(R_1, R_2, T_1, T_2, \lambda_{\rm min}, 
\lambda_{\rm max})$ and nothing else.
The only random variable involved in the flux being received at the detector is 
$r$, so we can indeed write the flux received at the detector as in 
Eq.~\ref{eq:flux_at_detector}.

We can finally write out the probability density functions for the signal to 
noise ratios in the single and double-star cases by using the transformation 
rule for pdfs, and applying Eq.~\ref{eq:snr_ivory_tower}.
For single stars,
\begin{equation}
{\rm prob}(x_s) = \frac{3}{d_{\rm max, s}^3} c_s^{3/2} \delta^3 \left( A T_{\rm 
dur} N_{\rm tra}\right)^{3/2} x_s^{-4}.
\label{eq:prob_xs}
\end{equation}
Analogously for double stars,
\begin{equation}
{\rm prob}(x_d) = \frac{3}{d_{\rm max, d}^3} c_d^{3/2} (\mathcal{D}\delta)^3 
\left( A T_{\rm dur} N_{\rm tra}\right)^{3/2} x_d^{-4}.
\label{eq:prob_xd}
\end{equation}


\subsection{Number of detected planets}

Performing the integrals of Eq.~\ref{eq:completeness_short}, we get:

\begin{align}
N_{{\rm det},s} &= N_s \Gamma_{t,s} f_{s,{\rm S/N} > {\rm (S/N)_{min}}} \\
&= N_s \Gamma_{t,s} \frac{1}{d_{\rm max,s}^3} c_s^{3/2} \delta^3 \left(A 
T_{\rm dur} N_{\rm tra}\right)^{3/2} x_{\rm min}^{-3},
\end{align}
and
\begin{align}
N_{{\rm det},d} &= 2 N_d \Gamma_{t,d} f_{d,{\rm S/N} > {\rm (S/N)_{min}}} \\
&= 2 N_d \Gamma_{t,d} 
\frac{1}{d_{\rm max,d}^3} c_d^{3/2} (\mathcal{D}\delta)^3 \left(A T_{\rm dur} 
N_{\rm tra}\right)^{3/2} x_{\rm min}^{-3}.
\end{align}

Formally, the $f_{s,{\rm S/N} > {\rm (S/N)_{min}}}$ term should really be 
written as ${\rm min}(1, \ldots)$, where $(\ldots)$ is the given expression. 
This ensures that the fraction of planets above the signal to noise threshold 
is less than 1.

The number of detected planets $N_{\rm det}$ is the sum of the two previously 
written equations, and can be written
\begin{align}
N_{\rm det} = 
&\left( \frac{\delta}{x_{\rm min}} \right)^3 (A T_{\rm dur} N_{\rm tra})^{3/2} 
\nonumber\\
&\quad \times
\left[ N_s \Gamma_{t,s} \frac{c_s^{3/2}}{d_{\rm max,s}^3}  +
	   2 N_d \Gamma_{t,d} \frac{c_d^{3/2}}{d_{\rm max,d}^3}  \right].
\label{eq:N_det}
\end{align}
All terms can be input to a computer, and checked against a Monte Carlo 
simulation if desired.


\section{Astronomer A ignores binarity}
Astronomer A has never heard of binary star systems. What occurrence rate does 
he derive for planets of radius $R_p$ and period $P$?

The {\it total} occurrence rate (number of detections divided number of 
``stars'') for Astronomer A would be $N_{\rm det}/(N_s + N_d)$. However, even 
though Astronomer A does not know about binaries, the radii he derives for any 
planets in binary systems are too small, by a factor $\sqrt{\mathcal{D}}$.
The question is what occurrence rate is derived for planets of radius $R_p$ and 
period $P$.
The answer is thus
\begin{equation}
\Gamma_{\rm A,\ planets\ of\ R_p} = \frac{N_{\rm det,s}}{N_s + N_d}.
\end{equation}

This astronomer will also think there is a second population of planets, with 
radius $R_p \sqrt{\mathcal{D}}$, and will thus rush to Nature claiming to also
have derived a second occurrence rate,
\begin{equation}
\Gamma_{\rm A,\ planets\ of\ R_p \sqrt{\mathcal{D}}} = \frac{N_{\rm det,d}}{N_s 
+ N_d}.
\end{equation}


\section{Astronomer B counts host stars correctly}
Astronomer B can somehow account correctly for the ``2 for 1'' 
effect of binarity, \textit{i.e.} that the sample actually has more stars than 
astronomer A thought.

By the same token as above,
\begin{equation}
\Gamma_{\rm B,\ planets\ of\ R_p} = \frac{N_{\rm det,s}}{N_s + 2N_d},
\end{equation}
and
\begin{equation}
\Gamma_{\rm B,\ planets\ of\ R_p \sqrt{\mathcal{D}}} = \frac{N_{\rm det,d}}{N_s 
	+ 2N_d}.
\end{equation}



\section{Astronomer C counts host stars correctly and figures out diluted radii}
Astronomer C did high resolution imaging followup on every candidate, and 
correctly classifies the planetary radii.
Thus, she also knows which planets are in binary systems, and which are in 
single star systems.

She knows that the purported population of planets with radii $R_p 
\sqrt{\mathcal{D}}$ does not exist. All detected planets from this survey have 
radii $R_p$. She computes an occurrence rate
\begin{align}
\Gamma_{\rm C,\ planets\ of\ R_p} &= 
\frac{N_{\rm det,s} + N_{\rm det,d}}{N_s + 2N_d} \\
&= \frac{N_{\rm det}}{N_s + 2N_d},
\end{align}

the closest yet to the true rate (Sec.~\ref{sec:true_rate}).


\section{Astronomer D counts host stars correctly, figures out diluted radii, 
and accounts for completeness}

Astronomer D, just like Astronomer C, knows which detections were around 
binaries and the associated radius correction.

Astronomer D notes that the number of detected planets is not equal to the 
actual number of planets in the sample.
They must account for completeness.
They do injection recovery, and derive independent estimates for their 
completeness functions about single stars, $f_{s,x>x_{\rm min}} \equiv f_s$ and 
about double star systems, $f_{d,x>x_{\rm min}} \equiv f_d$.
With this knowledge in hand, they compute the respective single and binary 
occurrence rates
\begin{equation}
\Gamma_{t,s} = \frac{N_{\rm det,s}}{N_s f_s},
\end{equation}
\begin{equation}
\Gamma_{t,d} = \frac{N_{\rm det,d}}{2 N_d f_d}.
\end{equation}

With these in hand, they derive the overall occurrence rate
\begin{align}
\Gamma_{\rm D,\ planets\ of\ R_p} 
&= \frac{N_{\rm det,s}/f_s + N_{\rm det,d}/f_d  }{N_s + 2 N_d} \\
&= \frac{\Gamma_{t,s} N_s + 2\Gamma_{t,d} N_d}{N_s + 2 N_d}.
\end{align}

Although we admit it's been a bit of a slog, it happens that Astronomer D's 
occurrence rate is the true occurrence rate (cf. Eq.~\ref{eq:true_occ}).

All it takes is $\approx$ a full semester at Keck, a system-by-system analysis, 
and perfect understanding of the completeness of the detection efficiency for 
single and double star systems.


\section{Numerical verification}

As a check on the preceding analytic development, we implemented a Monte Carlo 
simulation of this idealized transit survey.
To run the survey, we defined the instrument specifications (detector area and 
transmission function), the stellar population (binary fraction, total number 
density of a given stellar class, binary light ratio, fixed stellar 
properties), the planet population (fixed planet radius, period, and occurrence 
rate about single and binary stars), and finally the survey parameters 
(observing baseline, minimum SNR for ``detection'').
We then randomly drew star positions, randomly assigned planets to stars in 
single and binary systems, and computed the resulting signal to noise 
(Eq.~\ref{eq:snr_ivory_tower}) with which 
the transits would be observed.
As in the preceding analytics\footnote{TODO could be to generalize away from 
this}, we assumed ``twin'' binaries (same stellar radii, same effective 
temperature, and dilution does not depend on which stellar binary is the 
``target'').

The results are shown in Fig.~\ref{fig:snr_dist_analytic_v_numeric}, and 
indicate that the analytic probability distribution functions 
Eqs.~\ref{eq:prob_xs},~\ref{eq:prob_xd}, and the number of 
detections (Eq.~\ref{eq:N_det}) are correct.

A point evident in Fig.~\ref{fig:snr_dist_analytic_v_numeric} is that, for 
fixed planet parameters, and fixed stellar parameters ($R_\star, L_\star$, and 
distance $r$) the SNR distribution for planets in binaries is poorer than that 
of planets in single star systems.
We can see analytically that this can be written only as function of the 
binary light ratio:
\begin{equation}
\frac{{\rm prob}(x_d)}{{\rm prob}(x_s)} =(1 + \gamma_R)^{-1}.
\end{equation}
Deriving this simple form requires noting that the ratios of the 
bandpass-specific number luminosities (Eq.~\ref{eq:c_i}) is equal to the ratio 
of the bandpass-specific energy luminosities (otherwise a term with $c_s/c_d$ 
must be included).

\begin{figure}[!t]
	\begin{center}
		\includegraphics[scale=0.4]{figures/snr_distribution.pdf}
	\end{center}
	\caption{Comparison of analytic and numeric probability 
	density functions of the SNR in an idealized transit survey.
	The analytic lines are Eqs.~\ref{eq:prob_xs} and~\ref{eq:prob_xd} for the 
	planet populations orbiting single and binary stars. The underlying stepped 
	histogram is output from Monte Carlo simulations. Poisson noise leads to 
	a small deviation at the faint and bright limits, but the numerics and 
	analytics otherwise agree.
		 }
	\label{fig:snr_dist_analytic_v_numeric}
\end{figure}


\section{Representative numbers for a few cases}

\subsection{Twin binaries: if we ignore binarity, for what fraction of 
detections do we misclassify the radii?}
 
Ignoring binarity, we will detect $N_{\rm det,s}$ planets around single stars, 
and $N_{\rm det,d}$ planets around double stars. The latter set will be assumed 
to have radii $R_p \sqrt{\mathcal{D}}$.
The fraction of detections with misclassified radii can then be written
\begin{equation}
\frac{N_{\rm det,d}}{N_{\rm det,s} + N_{\rm det,d}} = \frac{1}{1+\alpha},
\end{equation}
for
\begin{equation}
\alpha \equiv \frac{1}{2({\rm BF})} (1+\gamma_R)^{3/2} 
\frac{\Gamma_{t,s}}{\Gamma_{t,d}}.
\end{equation}

For the nominal G2V dwarf case of ${\rm BF}=0.45$, twin binaries with equal 
occurrence rates this produces a misclassification rate of $24\%$, in agreement 
with Fig.~\ref{fig:snr_dist_analytic_v_numeric}.

\subsection{Twin binaries: if we ignore binarity, how wrong is our occurrence 
rate for planets of radius $R_p$?}

This is almost simply asking ``what is the relative difference between the 
occurrence rates derived by Astronomers D and A for planets of radius $R_p$?''
However, in the more realistic case, Astronomer A also has derived a 
completeness, which we assume is the same as for Astronomer D in the single 
star case.
So Astronomer A now misclassifies planetary radii, and miscounts the total 
number of stars, but knows his completeness for single stars.
Astronomer D corrects all these errors.

For brevity, write $\Gamma_{\rm A,\ planets\ of\ R_p} = \Gamma_{\rm A,R_p}$, 
and similarly for ${\rm D}$.
Then the relative difference between the two occurrence rates is

\begin{align}
\left|\frac{\Gamma_{\rm A,R_p} - \Gamma_{\rm D,R_p}}
		   {\Gamma_{\rm A,R_p}} \right| &=
\left| 1 - \frac{\Gamma_{\rm D,R_p}}{\Gamma_{\rm A,R_p}}\right| \\
&=
\left| 1 - \left(\frac{\Gamma_{t,s}N_s + 2\Gamma_{t,d}N_d}{N_s + 2N_d}  \cdot 
\frac{N_s + N_d}{\Gamma_{t,s} N_s }\right)\right| \\
&= 
\left|
1 - \frac{(1 + 2 \beta \Gamma_{t,d}/\Gamma_{t,s})(1 + \beta)}{(1+2\beta)}
\right|,
\end{align}
for
\begin{equation}
\beta \equiv N_d/N_s = {\rm BF} \times (1+\gamma_R)^{3/2}.
\end{equation}

For the nominal G2V dwarf case of ${\rm BF}=0.45$ with twin binaries ($\gamma_R 
= 1$) and $\Gamma_{t,d} = \Gamma_{t,s}$ this gives a relative error of $127\%$.
For instance, in the numerical simulation corresponding to 
Fig.~\ref{fig:snr_dist_analytic_v_numeric}, Astronomer A finds $\Gamma_{\rm 
A,R_p} = 0.22$, while Astronomer 
D derives the true (input) occurrence rate of $\Gamma_{\rm D,R_p} = 0.5$.


\newpage

%\begin{thebibliography}{}


%\end{thebibliography}

\end{document}
